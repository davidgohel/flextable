% Options for packages loaded elsewhere
% Options for packages loaded elsewhere
\PassOptionsToPackage{unicode}{hyperref}
\PassOptionsToPackage{hyphens}{url}
\PassOptionsToPackage{dvipsnames,svgnames,x11names}{xcolor}
%
\documentclass[
  letterpaper,
  DIV=11,
  numbers=noendperiod]{scrartcl}
\usepackage{xcolor}
\usepackage{amsmath,amssymb}
\setcounter{secnumdepth}{-\maxdimen} % remove section numbering
\usepackage{iftex}
\ifPDFTeX
  \usepackage[T1]{fontenc}
  \usepackage[utf8]{inputenc}
  \usepackage{textcomp} % provide euro and other symbols
\else % if luatex or xetex
  \usepackage{unicode-math} % this also loads fontspec
  \defaultfontfeatures{Scale=MatchLowercase}
  \defaultfontfeatures[\rmfamily]{Ligatures=TeX,Scale=1}
\fi
\usepackage{lmodern}
\ifPDFTeX\else
  % xetex/luatex font selection
\fi
% Use upquote if available, for straight quotes in verbatim environments
\IfFileExists{upquote.sty}{\usepackage{upquote}}{}
\IfFileExists{microtype.sty}{% use microtype if available
  \usepackage[]{microtype}
  \UseMicrotypeSet[protrusion]{basicmath} % disable protrusion for tt fonts
}{}
\makeatletter
\@ifundefined{KOMAClassName}{% if non-KOMA class
  \IfFileExists{parskip.sty}{%
    \usepackage{parskip}
  }{% else
    \setlength{\parindent}{0pt}
    \setlength{\parskip}{6pt plus 2pt minus 1pt}}
}{% if KOMA class
  \KOMAoptions{parskip=half}}
\makeatother
% Make \paragraph and \subparagraph free-standing
\makeatletter
\ifx\paragraph\undefined\else
  \let\oldparagraph\paragraph
  \renewcommand{\paragraph}{
    \@ifstar
      \xxxParagraphStar
      \xxxParagraphNoStar
  }
  \newcommand{\xxxParagraphStar}[1]{\oldparagraph*{#1}\mbox{}}
  \newcommand{\xxxParagraphNoStar}[1]{\oldparagraph{#1}\mbox{}}
\fi
\ifx\subparagraph\undefined\else
  \let\oldsubparagraph\subparagraph
  \renewcommand{\subparagraph}{
    \@ifstar
      \xxxSubParagraphStar
      \xxxSubParagraphNoStar
  }
  \newcommand{\xxxSubParagraphStar}[1]{\oldsubparagraph*{#1}\mbox{}}
  \newcommand{\xxxSubParagraphNoStar}[1]{\oldsubparagraph{#1}\mbox{}}
\fi
\makeatother

\usepackage{color}
\usepackage{fancyvrb}
\newcommand{\VerbBar}{|}
\newcommand{\VERB}{\Verb[commandchars=\\\{\}]}
\DefineVerbatimEnvironment{Highlighting}{Verbatim}{commandchars=\\\{\}}
% Add ',fontsize=\small' for more characters per line
\usepackage{framed}
\definecolor{shadecolor}{RGB}{241,243,245}
\newenvironment{Shaded}{\begin{snugshade}}{\end{snugshade}}
\newcommand{\AlertTok}[1]{\textcolor[rgb]{0.68,0.00,0.00}{#1}}
\newcommand{\AnnotationTok}[1]{\textcolor[rgb]{0.37,0.37,0.37}{#1}}
\newcommand{\AttributeTok}[1]{\textcolor[rgb]{0.40,0.45,0.13}{#1}}
\newcommand{\BaseNTok}[1]{\textcolor[rgb]{0.68,0.00,0.00}{#1}}
\newcommand{\BuiltInTok}[1]{\textcolor[rgb]{0.00,0.23,0.31}{#1}}
\newcommand{\CharTok}[1]{\textcolor[rgb]{0.13,0.47,0.30}{#1}}
\newcommand{\CommentTok}[1]{\textcolor[rgb]{0.37,0.37,0.37}{#1}}
\newcommand{\CommentVarTok}[1]{\textcolor[rgb]{0.37,0.37,0.37}{\textit{#1}}}
\newcommand{\ConstantTok}[1]{\textcolor[rgb]{0.56,0.35,0.01}{#1}}
\newcommand{\ControlFlowTok}[1]{\textcolor[rgb]{0.00,0.23,0.31}{\textbf{#1}}}
\newcommand{\DataTypeTok}[1]{\textcolor[rgb]{0.68,0.00,0.00}{#1}}
\newcommand{\DecValTok}[1]{\textcolor[rgb]{0.68,0.00,0.00}{#1}}
\newcommand{\DocumentationTok}[1]{\textcolor[rgb]{0.37,0.37,0.37}{\textit{#1}}}
\newcommand{\ErrorTok}[1]{\textcolor[rgb]{0.68,0.00,0.00}{#1}}
\newcommand{\ExtensionTok}[1]{\textcolor[rgb]{0.00,0.23,0.31}{#1}}
\newcommand{\FloatTok}[1]{\textcolor[rgb]{0.68,0.00,0.00}{#1}}
\newcommand{\FunctionTok}[1]{\textcolor[rgb]{0.28,0.35,0.67}{#1}}
\newcommand{\ImportTok}[1]{\textcolor[rgb]{0.00,0.46,0.62}{#1}}
\newcommand{\InformationTok}[1]{\textcolor[rgb]{0.37,0.37,0.37}{#1}}
\newcommand{\KeywordTok}[1]{\textcolor[rgb]{0.00,0.23,0.31}{\textbf{#1}}}
\newcommand{\NormalTok}[1]{\textcolor[rgb]{0.00,0.23,0.31}{#1}}
\newcommand{\OperatorTok}[1]{\textcolor[rgb]{0.37,0.37,0.37}{#1}}
\newcommand{\OtherTok}[1]{\textcolor[rgb]{0.00,0.23,0.31}{#1}}
\newcommand{\PreprocessorTok}[1]{\textcolor[rgb]{0.68,0.00,0.00}{#1}}
\newcommand{\RegionMarkerTok}[1]{\textcolor[rgb]{0.00,0.23,0.31}{#1}}
\newcommand{\SpecialCharTok}[1]{\textcolor[rgb]{0.37,0.37,0.37}{#1}}
\newcommand{\SpecialStringTok}[1]{\textcolor[rgb]{0.13,0.47,0.30}{#1}}
\newcommand{\StringTok}[1]{\textcolor[rgb]{0.13,0.47,0.30}{#1}}
\newcommand{\VariableTok}[1]{\textcolor[rgb]{0.07,0.07,0.07}{#1}}
\newcommand{\VerbatimStringTok}[1]{\textcolor[rgb]{0.13,0.47,0.30}{#1}}
\newcommand{\WarningTok}[1]{\textcolor[rgb]{0.37,0.37,0.37}{\textit{#1}}}

\usepackage{longtable,booktabs,array}
\usepackage{calc} % for calculating minipage widths
% Correct order of tables after \paragraph or \subparagraph
\usepackage{etoolbox}
\makeatletter
\patchcmd\longtable{\par}{\if@noskipsec\mbox{}\fi\par}{}{}
\makeatother
% Allow footnotes in longtable head/foot
\IfFileExists{footnotehyper.sty}{\usepackage{footnotehyper}}{\usepackage{footnote}}
\makesavenoteenv{longtable}
\usepackage{graphicx}
\makeatletter
\newsavebox\pandoc@box
\newcommand*\pandocbounded[1]{% scales image to fit in text height/width
  \sbox\pandoc@box{#1}%
  \Gscale@div\@tempa{\textheight}{\dimexpr\ht\pandoc@box+\dp\pandoc@box\relax}%
  \Gscale@div\@tempb{\linewidth}{\wd\pandoc@box}%
  \ifdim\@tempb\p@<\@tempa\p@\let\@tempa\@tempb\fi% select the smaller of both
  \ifdim\@tempa\p@<\p@\scalebox{\@tempa}{\usebox\pandoc@box}%
  \else\usebox{\pandoc@box}%
  \fi%
}
% Set default figure placement to htbp
\def\fps@figure{htbp}
\makeatother





\setlength{\emergencystretch}{3em} % prevent overfull lines

\providecommand{\tightlist}{%
  \setlength{\itemsep}{0pt}\setlength{\parskip}{0pt}}



 


\usepackage{multirow}
\usepackage{multicol}
\usepackage{colortbl}
\usepackage{hhline}
\newlength\Oldarrayrulewidth
\newlength\Oldtabcolsep
\usepackage{longtable}
\usepackage{array}
\usepackage{hyperref}
\usepackage{float}
\usepackage{wrapfig}
\KOMAoption{captions}{tableheading}
\makeatletter
\@ifpackageloaded{caption}{}{\usepackage{caption}}
\AtBeginDocument{%
\ifdefined\contentsname
  \renewcommand*\contentsname{Table of contents}
\else
  \newcommand\contentsname{Table of contents}
\fi
\ifdefined\listfigurename
  \renewcommand*\listfigurename{List of Figures}
\else
  \newcommand\listfigurename{List of Figures}
\fi
\ifdefined\listtablename
  \renewcommand*\listtablename{List of Tables}
\else
  \newcommand\listtablename{List of Tables}
\fi
\ifdefined\figurename
  \renewcommand*\figurename{Figure}
\else
  \newcommand\figurename{Figure}
\fi
\ifdefined\tablename
  \renewcommand*\tablename{Table}
\else
  \newcommand\tablename{Table}
\fi
}
\@ifpackageloaded{float}{}{\usepackage{float}}
\floatstyle{ruled}
\@ifundefined{c@chapter}{\newfloat{codelisting}{h}{lop}}{\newfloat{codelisting}{h}{lop}[chapter]}
\floatname{codelisting}{Listing}
\newcommand*\listoflistings{\listof{codelisting}{List of Listings}}
\makeatother
\makeatletter
\makeatother
\makeatletter
\@ifpackageloaded{caption}{}{\usepackage{caption}}
\@ifpackageloaded{subcaption}{}{\usepackage{subcaption}}
\makeatother
\usepackage{bookmark}
\IfFileExists{xurl.sty}{\usepackage{xurl}}{} % add URL line breaks if available
\urlstyle{same}
\hypersetup{
  pdftitle={Pdflatex not working with flextable},
  colorlinks=true,
  linkcolor={blue},
  filecolor={Maroon},
  citecolor={Blue},
  urlcolor={Blue},
  pdfcreator={LaTeX via pandoc}}


\title{Pdflatex not working with flextable}
\author{}
\date{}
\begin{document}
\maketitle


\begin{Shaded}
\begin{Highlighting}[]
\FunctionTok{library}\NormalTok{(flextable)}


\CommentTok{\# different ways to access quarto metadata {-}{-}{-}{-}}
\DocumentationTok{\#\# Many things listed in quarto\_metadata, like default options, but not the current options used in the yaml}
\NormalTok{quarto\_metadata }\OtherTok{\textless{}{-}}\NormalTok{ knitr}\SpecialCharTok{::}\NormalTok{opts\_current}\SpecialCharTok{$}\FunctionTok{get}\NormalTok{()}
\FunctionTok{saveRDS}\NormalTok{(quarto\_metadata, }\StringTok{"./quarto\_metadata.rds"}\NormalTok{)}

\DocumentationTok{\#\# Returns an empty string, maybe not using it properly}
\NormalTok{quarto\_metadata\_second\_way }\OtherTok{\textless{}{-}}\NormalTok{ jsonlite}\SpecialCharTok{::}\FunctionTok{read\_json}\NormalTok{(}
  \FunctionTok{Sys.getenv}\NormalTok{(}\StringTok{"QUARTO\_EXECUTE\_INFO"}\NormalTok{))}
\CommentTok{\# two ways to reach the pdf{-}engine}
\FunctionTok{all.equal}\NormalTok{(quarto\_metadata\_second\_way}\SpecialCharTok{$}\NormalTok{format}\SpecialCharTok{$}\NormalTok{pandoc}\SpecialCharTok{$}\StringTok{\textasciigrave{}}\AttributeTok{pdf{-}engine}\StringTok{\textasciigrave{}}\NormalTok{, }
\NormalTok{          quarto\_metadata\_second\_way}\SpecialCharTok{$}\NormalTok{format}\SpecialCharTok{$}\NormalTok{metadata}\SpecialCharTok{$}\NormalTok{format}\SpecialCharTok{$}\NormalTok{pdf}\SpecialCharTok{$}\StringTok{\textasciigrave{}}\AttributeTok{pdf{-}engine}\StringTok{\textasciigrave{}}\NormalTok{)}
\end{Highlighting}
\end{Shaded}

\begin{verbatim}
[1] TRUE
\end{verbatim}

\begin{Shaded}
\begin{Highlighting}[]
\FunctionTok{saveRDS}\NormalTok{(quarto\_metadata\_second\_way, }\StringTok{"./quarto\_metadata\_second\_way.rds"}\NormalTok{)}

\DocumentationTok{\#\# This command returns exactly the yaml of the header, great!! :{-})}
\NormalTok{rmarkdown\_metadata }\OtherTok{\textless{}{-}}\NormalTok{ rmarkdown}\SpecialCharTok{::}\NormalTok{metadata}
\FunctionTok{saveRDS}\NormalTok{(rmarkdown\_metadata, }\StringTok{"./rmarkdown\_metadata.rds"}\NormalTok{)}

\DocumentationTok{\#\# Returns only the output file}
\NormalTok{rmarkdown\_pandoc\_args }\OtherTok{\textless{}{-}}\NormalTok{ knitr}\SpecialCharTok{::}\NormalTok{opts\_knit}\SpecialCharTok{$}\FunctionTok{get}\NormalTok{(}\StringTok{"rmarkdown.pandoc.args"}\NormalTok{)}
\FunctionTok{saveRDS}\NormalTok{(rmarkdown\_pandoc\_args, }\StringTok{"./rmarkdown\_pandoc\_args.rds"}\NormalTok{)}

\CommentTok{\# rmarkdown::render("./tests/testthat/qmd/use{-}printer{-}with{-}pdflatex.qmd",}
\CommentTok{\#                   output\_format = c("pdf\_document"),}
\CommentTok{\#                   output\_file = "C:/Users/basti/OneDrive/04. BNCL (Post Doc)/Open{-}Source Development/flextable/tests/testthat/qmd/use{-}printer{-}with{-}pdflatex.qmd")}


\CommentTok{\# quarto::quarto\_render("./tests/testthat/qmd/use{-}printer{-}with{-}pdflatex.qmd")}
\end{Highlighting}
\end{Shaded}

\begin{itemize}
\tightlist
\item
  \texttt{flextable} with Equations, see Table~\ref{tbl-flextable}:
\end{itemize}

\begin{Shaded}
\begin{Highlighting}[]
\NormalTok{eqs\_flextable }\OtherTok{\textless{}{-}} \FunctionTok{c}\NormalTok{(}
    \StringTok{"(ax\^{}2 + bx + c = 0)"}\NormalTok{,}
    \StringTok{"a }\SpecialCharTok{\textbackslash{}\textbackslash{}}\StringTok{ne 0"}\NormalTok{,}
    \StringTok{"x = \{{-}b }\SpecialCharTok{\textbackslash{}\textbackslash{}}\StringTok{pm }\SpecialCharTok{\textbackslash{}\textbackslash{}}\StringTok{sqrt\{b\^{}2{-}4ac\} }\SpecialCharTok{\textbackslash{}\textbackslash{}}\StringTok{over 2a\}"}\NormalTok{)}
\NormalTok{df }\OtherTok{\textless{}{-}}\NormalTok{ tibble}\SpecialCharTok{::}\FunctionTok{tibble}\NormalTok{(}\StringTok{\textasciigrave{}}\AttributeTok{Y }\SpecialCharTok{\textbackslash{}\textbackslash{}}\AttributeTok{sim W}\StringTok{\textasciigrave{}} \OtherTok{=}\NormalTok{ eqs\_flextable)}


\NormalTok{ft }\OtherTok{\textless{}{-}} \FunctionTok{flextable}\NormalTok{(df) }\SpecialCharTok{|\textgreater{}}
  \FunctionTok{compose}\NormalTok{(}\AttributeTok{j =} \DecValTok{1}\NormalTok{, }\AttributeTok{part =} \StringTok{"header"}\NormalTok{,}
          \AttributeTok{value =} \FunctionTok{as\_paragraph}\NormalTok{(}\FunctionTok{as\_equation}\NormalTok{(}\StringTok{\textasciigrave{}}\AttributeTok{Y }\SpecialCharTok{\textbackslash{}\textbackslash{}}\AttributeTok{sim W}\StringTok{\textasciigrave{}}\NormalTok{, }\AttributeTok{width =} \DecValTok{2}\NormalTok{, }\AttributeTok{height =}\NormalTok{ .}\DecValTok{5}\NormalTok{)),}
          \AttributeTok{use\_dot =} \ConstantTok{TRUE}\NormalTok{) }\SpecialCharTok{|\textgreater{}}
  \FunctionTok{compose}\NormalTok{(}\AttributeTok{j =} \DecValTok{1}\NormalTok{, }\AttributeTok{part =} \StringTok{"body"}\NormalTok{,}
          \AttributeTok{value =} \FunctionTok{as\_paragraph}\NormalTok{(}\FunctionTok{as\_equation}\NormalTok{(}\StringTok{\textasciigrave{}}\AttributeTok{Y }\SpecialCharTok{\textbackslash{}\textbackslash{}}\AttributeTok{sim W}\StringTok{\textasciigrave{}}\NormalTok{, }\AttributeTok{width =} \DecValTok{2}\NormalTok{, }\AttributeTok{height =}\NormalTok{ .}\DecValTok{5}\NormalTok{))) }\SpecialCharTok{|\textgreater{}}
  \FunctionTok{align}\NormalTok{(}\AttributeTok{align =} \StringTok{"center"}\NormalTok{, }\AttributeTok{part =} \StringTok{"all"}\NormalTok{)}

\NormalTok{ft}
\end{Highlighting}
\end{Shaded}

\begin{verbatim}
Warning: fonts used in `flextable` are ignored because the `pdflatex` engine is
used and not `xelatex` or `lualatex`. You can avoid this warning by using the
`set_flextable_defaults(fonts_ignore=TRUE)` command or use a compatible engine
by defining `latex_engine: xelatex` in the YAML header of the R Markdown
document, orlatex-engine: xelatex (or lualatex) in a Quarto document.
\end{verbatim}

\global\setlength{\Oldarrayrulewidth}{\arrayrulewidth}

\global\setlength{\Oldtabcolsep}{\tabcolsep}

\setlength{\tabcolsep}{2pt}

\renewcommand*{\arraystretch}{1.5}



\providecommand{\ascline}[3]{\noalign{\global\arrayrulewidth #1}\arrayrulecolor[HTML]{#2}\cline{#3}}

\begin{longtable}[c]{|p{0.75in}}

\caption{\label{tbl-flextable}}

\tabularnewline

\ascline{1.5pt}{666666}{1-1}

\multicolumn{1}{>{\centering}m{\dimexpr 0.75in+0\tabcolsep}}{\textcolor[HTML]{000000}{\fontsize{11}{11}\selectfont{$Y \sim W$}}} \\

\ascline{1.5pt}{666666}{1-1}\endfirsthead 

\ascline{1.5pt}{666666}{1-1}

\multicolumn{1}{>{\centering}m{\dimexpr 0.75in+0\tabcolsep}}{\textcolor[HTML]{000000}{\fontsize{11}{11}\selectfont{$Y \sim W$}}} \\

\ascline{1.5pt}{666666}{1-1}\endhead



\multicolumn{1}{>{\centering}m{\dimexpr 0.75in+0\tabcolsep}}{\textcolor[HTML]{000000}{\fontsize{11}{11}\selectfont{$(ax^2 + bx + c = 0)$}}} \\





\multicolumn{1}{>{\centering}m{\dimexpr 0.75in+0\tabcolsep}}{\textcolor[HTML]{000000}{\fontsize{11}{11}\selectfont{$a \ne 0$}}} \\





\multicolumn{1}{>{\centering}m{\dimexpr 0.75in+0\tabcolsep}}{\textcolor[HTML]{000000}{\fontsize{11}{11}\selectfont{$x = {-b \pm \sqrt{b^2-4ac} \over 2a}$}}} \\

\ascline{1.5pt}{666666}{1-1}


\end{longtable}

\arrayrulecolor[HTML]{000000}

\global\setlength{\arrayrulewidth}{\Oldarrayrulewidth}

\global\setlength{\tabcolsep}{\Oldtabcolsep}

\renewcommand*{\arraystretch}{1}




\end{document}
